%%%%%%%%%%%%%%%%%%%%%%%%%%%%%%%%%%%%%%%%%%%%%%%%%%%%%%%%%%%%%%%%%%% 
%                                                                 %
%                            ABSTRACT                             %
%                                                                 %
%%%%%%%%%%%%%%%%%%%%%%%%%%%%%%%%%%%%%%%%%%%%%%%%%%%%%%%%%%%%%%%%%%% 
 
\specialhead{ABSTRACT}


 
 
\begin{comment}
 As computational power has continued to increase, and sensors have become more accurate, thecorresponding advent of systems that are at once cognitive and immersive has arrived. Thesecog-nitive and immersive systems(CAISs) fall squarely into the intersection of AI with HCI/HRI: suchsystems interact with and assist the human agents that enter them, in no small part because suchsystems are infused with AI able to understand and reason about these humans and their knowl-edge, beliefs, goals, communications, plans, etc.  We herein explain our approach to engineeringCAISs.  We emphasize the capacity of a CAIS to develop and reason over a “theory of the mind”of its human partners. This capacity entails that the AI in question has a sophisticated model of thebeliefs, knowledge, goals, desires, emotions, etc. of these humans. To accomplish this engineering,a formal framework of very high expressivity is needed. In our case, this framework is acognitiveevent calculus, a particular kind of quantified multi-operator modal logic, and a matching high-expressivity automated reasoner and planner.  To explain,  advance,  and to a degree validate ourapproach, we show that a calculus of this type satisfies a set of formal requirements, and can enablea CAIS to understand a psychologically tricky scenario couched in what we call thecognitive-polysolid framework. We also show formally that a room that satisfies these requirements can havea useful property we termexpectation of usefulness.  This sub-class ofcognitive microworldsin-cludes machinery able to represent and plan over not merely blocks and actions (such as seen in the“blocks worlds” of old), but also over agents and their mental attitudes about both other agents andinanimate objects.
\end{comment}

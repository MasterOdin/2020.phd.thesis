\section{Cognitive Blockworld}

We now introduce the \emph{cognitive-polysolid framework} (CPF), a
class of problems that we use for our experiments. In these problems,
both human agents and polysolids are represented and questions concern the
cognitive states of the humans. For our purposes,
we classify our polysolids as regular 3D shapes, such as cubes, cylinders, spheres, etc.
that do not contain any holes or gaps in them. Using
the framework, we can generate a \emph{cognitive-polysolid world
instantiation}, where an instantiation has some number of polysolids,
declares how they can be moved, and specifies any agents and their possible
beliefs or knowledge about the polysolids and other agents.

CPF subsumes the familiar ``blocks world,'' described for instance in
\cite{nilsson_principles_1982}, which has long been used for reasoning and
planning tasks.  The framework gives us both a physical and cognitive
domain unlike the purely physical blocks world domain (The formal
logic used in \cite{nilsson_principles_1982} is purely extensional, as it is
simply first-order logic).  Since the physical complexities of blocks
world problems have been well explored
\cite{gupta_complexity_1992,slaney_blocks_2001}, we emphasize the cognitive extensions
of it.

The cognitive-polysolid world instantiation we focus on contains some finite number
of cubes/blocks and a table large enough to hold all of them.  Each block is
\texttt{on} one other object; where that object can be another block or the
table.  A block is said to be \texttt{clear} if there is no block that
is on top of it.  To move the blocks, an agent can either
\texttt{stack} (placing a block on the table on top of another block)
or \texttt{unstack} (taking a block that is on top of another block
and placing it on the table).  Before stacking a block onto another, both need
to be clear; when unstacking, the top block must be clear beforehand.
After stacking the blocks, the bottom block is then not clear, and
after unstacking, it is then now clear.

\vspace{-0.03in}
  \begin{equation*}
    \begin{aligned}\
      &\Surface \sqsubset \Object \\
      &\Block \sqsubset \Surface \\
      &\ctable: \Surface \\
      &\on: \Block \times \Surface \rightarrow \Fluent
             \end{aligned} \hspace{30pt}
 \begin{aligned}
      &\clear: \Block \rightarrow \Fluent \\
      &\goal: \Boolean \times \Number \rightarrow \Boolean \\
      &\stack: \Block \times \Block \rightarrow \ActionType \\
      &\unstack: \Block \times \Block \rightarrow \ActionType
    \end{aligned}
  \end{equation*}
\vspace{-0.05in}

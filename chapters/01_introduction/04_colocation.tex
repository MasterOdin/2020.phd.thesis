\subsection{On Colocation vs Distributed Usage}

Before continuing, we would like to take a moment to discuss the beliefs of
colocation vs distributed usage of these sorts of technologies. Within this
work, we present, and focus on, a vision of how such systems can be applied and
help with colocated participants. While we do believe that elements of this work
can be immediately applied to a digital domain, as noted by
Greenberg~\cite{greenberg_implications_2016}, the "we-awareness" that is
generated via face-to-face does not easily and quickly translate over to the
same digital sense. Take for example pointing to something on a screen. With two
people standing next to each other, and one pointing at a shared screen, we can
make reasonable assumptions on the cognitive states of agents, what they're
perceiving, and also of what they might believe about each other and what each
other perceives. Moving to the digital realm, these concepts cannot be
immediately recast as is, as one must first design the necessary mechanisms to
capture that information, transmit it to the other participants, and to do so in
such a way that the many rich and subtle parts of that face-to-face interaction
are not lost. Given the complexity of that task, and the research area onto
itself it would require, we do not attempt to give the parallels here, focusing
just on the topic of handling the work in the colocated version only.

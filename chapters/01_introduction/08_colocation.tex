\section{On Colocation vs Distributed Usage}

Before continuing, we would like to take a moment to discuss the beliefs of
colocation vs distributed usage of these sorts of technologies. Within this
work, we present, and focus on, a vision of how such systems can be applied and
help with colocated participants. While we do believe that elements of this work
can be applied to a digital domain, doing so fundamentally changes the paradigm
of collaboration. When working together, we use the humans utilize different
levels of coupling, increasing or decreasing based on the extent of
communication that is required by the task at
hand~\cite{salvador_denver_1996,olson_distance_2000}. However, it is important
that coupling is not a static concept through a task, rather that it is
transitive, changing through the lifetime of a task~\cite{jakobsen_up_2014}.
Within this coupling, we see rise to a concept of
"we-awareness" amongst participants~\cite{greenberg_implications_2016}, as they
utilize some level implicit understanding of verbal and non-verbal
communications. These concepts do not easily transfer to distributed digital
systems wherein participants are no longer standing next to each other, rather
sharing a ``space'' through interfaces like video chat. Take for example
pointing to something on a screen. With two people standing next to each other,
and one pointing at a shared screen, we can make reasonable assumptions on the
cognitive states of agents, what they're perceiving, and also of what they might
believe about each other and what each other perceives.

Moving to the digital realm, these concepts cannot be
immediately recast as is, as one must first design the necessary mechanisms to
capture that information, transmit it to the other participants, and to do so in
such a way that the many rich and subtle parts of that face-to-face interaction
are not lost. Given the complexity of that task, and the research area onto
itself it would require, we do not attempt to give the parallels here, focusing
just on the topic of handling the work in the colocated version only.

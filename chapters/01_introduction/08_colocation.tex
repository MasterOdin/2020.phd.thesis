\section{On Co-location vs Distributed Usage}

Before continuing, we would like to take a moment to discuss the
some thoughts on handling co-location vs distributed usage for a
cognitive and immersive system.Within this work, we present, and focus
on, a vision of how such systems can be applied and help with co-located
participants. While we do believe that many of the elements of this work can be
applied to a digitally distributed domain, doing so fundamentally changes the
paradigm of collaboration. When working together, humans
%% MATT2:  "use the humans"?  Not sure what is meant here.  Pls
%% clarify, refine.  Thx.
%% S: typo, fixed.
utilize different levels of coupling, increasing or decreasing based
on the extent of communication that is required by the task at
hand~\cite{salvador_denver_1996,olson_distance_2000}. Jakobsen and
HornbÆk~\cite{jakobsen_up_2014} demonstrate that through the duration of
a task, humans may alter the level of coupling. Greenberg and
Gutwin~\cite{greenberg_implications_2016} term this concept of coupling,
wherein humans may communicate both verbally and non-verbally (e.g.
through body language) to each other as ``we-awareness''. These
concepts do not easily transfer to distributed
digital systems wherein participants are no longer standing next to
each other, rather sharing a ``space'' through interfaces like video
chat. Take for example pointing to something on a screen. With two
people standing next to each other, and one pointing at a shared
screen, we can make reasonable assumptions regarding the cognitive
states of agents, what they're perceiving, and also regarding what
they might believe about each other and what each other perceives.

Moving to the digital realm, these concepts cannot be immediately
recast as is, as one must first design the necessary mechanisms to
capture that information, transmit it to the other participants, and
to do so in such a way that the many rich and subtle parts of that
face-to-face interaction are not lost. Given the complexity of that
task, and the research area unto itself it would require, we do not
attempt to give the parallels here; we instead focus just on the topic
of handling the work in the co-located version only.

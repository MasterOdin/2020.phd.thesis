\subsection{Motivating Example}

In this section, we provide an overarching motivating example to help
illuminate the challenges we hope to solve within this work. To help further
ground the work, we present an example that could be commonplace within a joint
meeting space, and does not include a contrived setup.

Imagine that within a CAIS, there are three humans, Alvin, Betty, and Charlie,
who are all jointly working on a shared problem. Alvin and Betty are standing
next to each other, while Charlie stands apart. There is content on the screen
that relates to the problem. Betty looks at Alvin, then points at a particular
region of the screen. Alvin, seeing that Betty is looking at him, looks at her,
then follows to where she is pointing at the screen. Charlie is not looking at
either Alvin or Betty, nor at the region of screen that Betty is pointing at,
but instead at some other region of the screen. Betty gestures at the region of
the screen that she is pointing at, and the content changes in some way. Alvin,
who is following along the pointing and gesture, perceives both the gesture, and
the content change on the screen, while Charlie does not. Alvin then points to
the screen and makes a gesture, changing the content on the screen for a second
time. Betty perceives the pointing and gesture that Alvin does, while Charlie is
still oblivious to it. The time for this meeting concludes, and Betty might
reasonably ask "is everyone on the same page?". Neither Alvin or Betty are aware
that Charlie is not, while Charlie is unaware of the changed content. The
CAIS steps in, saying that no, Charlie is not on the same page, and that here
is a summary of the actions of Alvin and Betty for Charlie to catch him up,
covering the content that Alvin and Betty had both pointed at, as well as
the content that had changed on the screen.

From this example, we see that the CAIS must possess a number of sophisticated
pieces of machinery. For the physical space, the CAIS must possess the capacity
of knowing who is using the system, the physical actions that they
undertake, and what these agents might say. The CAIS then must possess the
capacity to display information to the human participants and speak to the
human participants. Finally, the CAIS must be able to connect the physical
actions of the humans, to that of the content that is being shown within the
space, such that it connects Betty gesturing at the screen with content under
that gesture.

\section{Requirements of CAIS}\label{ref:requirements_cais}

It should be noted that for a CAIS to be considered a truly
``intelligent'' room, it is not sufficient that the room be
intelligent about, for example, search queries over a domain $D$; the
room should also be intelligent about cognitive states of agents in
the room and their cognitive states towards $D$.

Despite there being a significant amount of work done in building
intelligent environments (of varying levels of intelligence;
\cite{coen_design_1998,brooks_intelligent_1997,chan_review_2008}),
there is no formalization of what constitutes an intelligent room and
what separates it from an intelligent agent.  Though
\cite{coen_design_1998} briefly differentiates an intelligent room
from ubiquitous computing based on the non-ubiquity of sensors in the
former, there is not any formal or rigorous discussion of what
separates an intelligent room from a mobile robot that roams around
the room with an array of sensors. As far as we are aware, this work
is novel in its act of providing a characterization of what separates
an intelligent room from an intelligent agent.

The requirements in question are cognitive in nature and exceed
intelligent rooms with sensors that can answer queries over simple
extensional data (e.g.\ a room that can answer financial queries such
as \textit{``Show me the number of companies with revenue over X?''}).
At a high-level, we require two conditions below hold:

\begin{itemize}
    \item $\mathcal{C}$ \emph{Cognitive}: A CAIS should be able to
      help agents with cognitive tasks and goals.  For instance, a
      system that simply aids in querying a domain $D$ is not
      cognitive in nature; a system that knowingly aids an agent in convincing
      another agent that some state-of-affairs holds in $D$ is
      considered cognitive. (Please see the appendix for more
      discussion of our usage of the term ``cognitive.'')
    \item $\mathcal{I}$ \emph{Immersive}: There should be some
      attribute or property of a CAIS that is non-localized and
      distinguished from agents in the room.  Moreover, this property
      should be \textbf{common knowledge}.\footnote{For our purposes,
        common knowledge as defined in Chapter~\ref{chap:formalizing}
        is that all agents know this property, and know that all other
        agents know it.}  (Note: this is not easily achievable with a
      physical robot, and this condition differentiates a CAIS from a
      cognitive agent.)
\end{itemize}

While we believe that $\mathcal{C}$ is fully realizable and is
achieved as part of this dissertation, $\mathcal{I}$ is certainly far
more ambitious, given the high level of sophistication necessary for
an implementation that satisfies this condition. As such, we posit
that for a true formalization to hold any water, it must stem from the
implementation and not vice versa. In this way, we can ground our
formalization in a working system that can properly showcase both the
formalization and our technology.

\begin{comment}
\footnote{This condition may not strictly be realizable,
        but the goal is to at a minimum build systems that approach
        this ideal condition.}
\end{comment}

\section{Related Work}\label{sec:cais_prior_work}

There is a rich background of work for the design of multi-modal systems, and 
more broadly the ``intelligent rooms'' in which these systems are situated. In
these systems, multi-modal input is seen as a necessity for more ``natural''
interactions that is afforded by traditional desktop environments.
Bolt~\cite{bolt_put-that-there:_1980} demonstrated a system that was capable
of combining pointing with ambiguous uses of ``that'' and ``there'' such as in
the sentence ``put that there''. This was done utilizing special hardware worn
by the user, and required the user to sit in a chair that the user could not
move from. Pentland~\cite{pentland_smart_1996} created a system that was
capable of tracking users' gestures and facial expressions for interaction
with a smart board. Brooks~\cite{brooks_intelligent_1997} created a full 
``intelligent room'' where users could more freely move around, but
that pointing required a user to go up to a touch the screen on which the
display is shown to incorporate gestures. This work is interesting in that
it was built such that users could interact with arbitrary webpages opened
in the Netscape browser on the display wall, utilizing deep hooks into the X
display server. While this worked well for their needs, it did limit the system
to only be able to run on certain OSes, and that maintaining it is complex due
to these deep hooks. The system was built on the
SodaBot~\cite{coen_sodabot:_1994} distributed control system, which aimed to 
overcome the inflexible nature of monothlic applications at the time. Similar
to Brooks' work, but taken to a further extreme, Roman et 
al.~\cite{roman_middleware_2002} created an entire OS middleware focused on running
intelligent rooms that also allowed users to interact with remotely. While very
powerful, maintaining such an approach is incredibly complex to implement and
maintain due to having its own file browser, video player, browser, etc.
However, these approaches were constrained by the technology at the time,
limiting the capabilities of speech supported, the usable workspace for a
user, and number of users supported. After a number of significant advances in 
the underlying technology, Carbini et al.~\cite{carbini_wizard_2006} 
demonstrate a system that allowed a single user to stand apart from the
display, and interact with it in multi-modal fashion, utilizing specially
calibrated stereo 3D cameras. Farrel et al.~\cite{farrell_symbiotic_2016}
proposes the concept of symbiotic AI that can be embedded into a room, and
can be seen as a close precursor to this work. Kephart et 
al.~\cite{kephart_embodied_2019} demonstrated usage of this system combined
with the ability to resolve ambiguous phrases with gestural information similar
to Bolt. Both of these supported multi-users but not across all modalities
simultaneously, such that two users could talk at the same time, but only one
could be pointing due to a hardware limitation. Allen et 
al.~\cite{allen_rensselaer_2019} demonstrate an immersive room in which multiple
users can gesture and speak in simultaneously, but utilizes a display framework
that is not easy to extend to non-game environments and cannot display certain
types of content, like web pages.

From the above list of work, we can compile a list of limitations from the above
work. The first is that many of the above, while supporting gestures such as
pointing, do so such that only supports one user, and through utilization of
sensors that require calibration, and in the cases of the cameras, limit wherein
the space a user can stand. Next, the content that is displayed and interacted
with in these environments are purpose built for a given domain. As part of this
process, the content is specially instrumented to support gestural information,
and as such does not support bringing in third-party content. As part of the
contributions of the dissertation, we aim for our implementation to avoid such
limitations, with the creation of novel technology, described below and in
subsequent chapters.
\section{Related Work}

There is a rich background of work for the design of multimodal systems, and more broadly the intelligent rooms
in which these systems are situated.

Historically, these systems have fallen into the category of ``smart rooms''
where multimodal input is much more of a necessity for more ``natural''
interaction than traditional desktop environments. In these multimodal
``smart rooms'', various efforts have been shown for picking up
where a user is pointing and using it in combination with
their voice to generate a command
\cite{bolt_put-that-there:_1980,carbini_wizard_2006,langner_multiple_2019,farrell_symbiotic_2016,kephart_embodied_2019}.
However, these systems hand-build their display layer and
content to embed the necessary instrumentation for combining speech
and gesture. Alternatively, \cite{coen_design_1998,brooks_intelligent_1997} demonstrate a system that
hooks into the X display server and could act on the content that was passed through
it from various applications, such as Netscape. While a very powerful approach, it
does not easily work across platforms, and that the security model of modern
applications such as Chrome renders the approach
increasingly impossible. Perhaps the most radical, \cite{roman_middleware_2002} propose an entire middleware infrastructure, similar to an OS,
to underpin these multimodal applications, but this approach
even more readily runs into the aforementioned bottleneck
in that all applications must be built from the ground up to
utilize this middleware, rendering it hard to reuse existing
content and displays.

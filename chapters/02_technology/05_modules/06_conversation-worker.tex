\subsection{conversation-worker}

The conversation-worker is responsible for taking in a sentence, usually from the transcript-worker
and converting into an tuple $<I, E>$ where $I$ is the intent and $E$ is our entities. An intent for our
purposes is an action to take within our system. The entities then are the elements that make up that action.
For example, given the sentence "Move block A onto block B", intent would be ``move'' with the entities being
``block A'' and ``block B''.

For our implementation of this component, we utilize the Watson Assistant service. Within this service,
we define intents, and then provide example sentences that would fit that intent to train the system.
This has the benefit of allowing the system the capacity to handle slightly differently worded sentences
than what we might expect, as well as gloss over spelling and grammatical errors that arise from the
transcription process. For entities, we utilize built-in system ones (such as numbers), as well as define
our own, where then entities are largely detected through from the input sentence using simple pattern matching.
For this system, we pre-build our worker with a list of intents and entities for a given domain, but also
allow on-the-fly training by users through mechanisms like Reagent, as explored in Chapter~\ref{chap:reagent}
\subsection{orchestrator}

The orchestrator is responsible for receiving the actionable intent from the user, and making
sure that all necessary entities are accounted for. Upon receiving an intent from an
upstream module, the orchestrator messages the executor via a RPC queue to determine if it
matches any known available inputs for the current system state. If there is one, as part of
the response, the executor will include the necessary inputs for executing that action. The orchestrator
then takes this response and compares the list of necessary inputs to the list of received inputs,
determining if there are any missing. If there are any missing inputs, the orchestrator is then
responsible for attempting to reconcile this, and if not possible, surfacing back to the user a
request for additional context to complete the command. This process is further explored in 
Chapter~\ref{chap:planning}.

\begin{comment}
For the cognitive condition $\mathcal{C}$ above, we have the following
concrete requirements that we implement in our system.  Later, we give
examples of these requirements in action.

  \begin{footnotesize}
    \begin{mdframed}[frametitle= Formal Requirements for $\mathcal{C}$ ,
      frametitlebackgroundcolor=gray!25, nobreak,
      linecolor=white,backgroundcolor=gray!10]
      Assume $\Gamma \vdash t < t + \Delta$
      \begin{enumerate}
      \item[$\mathbf{C}^f_1:$] It is common knowledge that, if an
        agent $x$ has a false belief, $\gamma$ informs the agent of
        the belief:\footnote{Please note that inference in \CEC\ is
          non-monotonic as it includes the event calculus, which is
          non-monotonic.  If an agent $a$ believes $\phi$ based on
          prior information, adding new information can cause the
          agent to not believe $\phi$.}
         \begin{equation*}
          \common\color{gray!80}\left(\color{black}t, \left[\begin{aligned}
                &  \believes(\gamma, t, \phi) \land \believes(\gamma, t,
                \believes\big(x, t, \lnot \phi)\big)
                \\ & \hspace{50pt} \rightarrow \\
                & \hspace{30pt}\says(\gamma, x, t + \Delta, \phi)
              \end{aligned}\right]\color{gray!80}\right)\color{black}
        \end{equation*}

      \item[$\mathbf{C}^f_2:$] It is common knowledge that, if an
        agent $x$ has a missing belief, $\gamma$ informs the agent of
        the belief:
        \begin{equation*}
          \common\color{gray!80}\left(\color{black}t,\left[ \begin{aligned}
                &  \believes(\gamma, t, \phi) \land \believes(\gamma, t,
                \lnot \believes\big(x, t, \phi)\big)
                \\ &  \hspace{50pt}  \rightarrow \\
                & \hspace{30pt}\says(\gamma, x, t + \Delta, \phi)
              \end{aligned}\right]\color{gray!80}\right)\color{black}
        \end{equation*} \end{enumerate}
    \end{mdframed}
  \end{footnotesize}

  For the immersive condition $\mathcal{I}$ above, we have the
  following conditions:

  \begin{footnotesize}
    \begin{mdframed}[frametitle= Formal Requirements for $\mathcal{I}$ ,
      frametitlebackgroundcolor=gray!25, nobreak,
      linecolor=white,backgroundcolor=gray!10]
      \begin{enumerate}
      \item[$\mathbf{I}^f_1:$] It is common knowledge that at any
        point in time $t$ an agent $x$, different from the CAIS system
        $\gamma$, can observe events or conditions (fluents) only in
        its vicinity.
        \begin{equation*}
          \begin{aligned}
            \common\Bigg(\forall x, t, f:& (x \not = \gamma) \rightarrow \Big[\perceives\big(x, t, \holds(f, t)\big) \rightarrow
            \holds(\vicinity(x, f), t)\Big]\Bigg)\\
            \common\Bigg(\forall x, t, e:& (x \not = \gamma) \rightarrow \Big[\perceives\big(x, t, \happens(e, t)\big) \rightarrow
            \holds(\vicinity(x, e), t)\Big]\Bigg)\\
          \end{aligned}
        \end{equation*}

      \item[$\mathbf{I}^f_2:$] It is common knowledge that actions
        performed by the agent are in its vicinity.
        \begin{equation*}
          \common\Big(\forall x, \alpha:  \holds\big(\vicinity(x, \action(a, \alpha)), t\big)\Big)
        \end{equation*}
      \item[$\mathbf{I}^f_3:$] It is common knowledge that all events
        and fluents are perceived by $\gamma$.  This is represented by
        the four conditions below:
        \begin{equation*}
          \begin{aligned}
       &  (i) \hspace{10pt}  \common\Bigg(\forall  t, f:  \Big[\holds(f, t) \leftrightarrow \perceives\big(\gamma, t, \holds(f, t)\big) \Big]\Bigg)\\
       & (ii)  \hspace{10pt}   \common\Bigg(\forall  t, e: \Big[\happens(e, t)
            \leftrightarrow \perceives\big(\gamma, t, \happens(e,
            t)\big) \Big]\Bigg)\\
        &   (iii)  \hspace{10pt}   \common\Bigg(\forall  t, f:  \Big[\lnot \holds(f, t)
            \leftrightarrow \perceives\big(\gamma, t, \lnot \holds(f, t)\big) \Big]\Bigg)\\
        &   (iv) \hspace{10pt}    \common\Bigg(\forall  t, e:\Big[\lnot \happens(e, t)
            \leftrightarrow \perceives\big(\gamma, t, \lnot \happens(e, t)\big) \Big]\Bigg)
          \end{aligned}
        \end{equation*}
      \end{enumerate}
    \end{mdframed}
  \end{footnotesize}

\end{comment}
  %%% Local Variables:
  %%% mode: latex
  %%% TeX-master: "main"
  %%% End:
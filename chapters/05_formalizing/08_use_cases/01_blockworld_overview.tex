\subsection{Block world}

We now introduce a formalization of the first of our two use-cases, the
``blocks world''~\cite{nilsson_principles_1982}. We specifically select
this use-case due to its historical nature within the planning domain.
This is due to two factors, (1) it is a relatively simple domain and is
easy to understand and describe in few sentences to a layman, and
(2) that the domain has been very well explored and its physical
complexities are very well
understood~\cite{gupta_complexity_1992,slaney_blocks_2001}. Within the
blocks world, there are two types of objects, a \Table\ and \Block.
On the \Table, there may be any number of $\Block$s.  Each block is
\texttt{on} one other object; where that object can be another block or the
table. A block is said to be \texttt{clear} if there is no block that
is on top of it. To move the blocks, an agent can either
\texttt{stack} (placing a block on the table on top of another block)
or \texttt{unstack} (taking a block that is on top of another block
and placing it on the table). Before stacking a block onto another, both need
to be clear; when unstacking, the top block must be clear beforehand.
After stacking the blocks, the bottom block is then not clear, and
after unstacking, it is then now clear.

\vspace{-0.03in}
  \begin{equation*}
    \begin{aligned}\
      &\Surface \sqsubset \Object \\
      &\Block \sqsubset \Surface \\
      &\Table \sqsubset \Surface \\
      &\on: \Block \times \Surface \rightarrow \Fluent
             \end{aligned} \hspace{30pt}
 \begin{aligned}
      &\clear: \Block \rightarrow \Fluent \\
      &\goal: \Boolean \times \Number \rightarrow \Boolean \\
      &\stack: \Block \times \Block \rightarrow \ActionType \\
      &\unstack: \Block \times \Block \rightarrow \ActionType
    \end{aligned}
  \end{equation*}

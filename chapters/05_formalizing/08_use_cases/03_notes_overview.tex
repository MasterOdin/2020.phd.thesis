\subsection{Sticky Notes}

In this section we present our digital sticky notes use-case and application. This
application provides a more real-world usage of our technologies versus the more
abstract Cognitive Block World, though is still simple enough to be readily
understandable. This use-case was designed to be as analogous to the traditional
pen-and-paper design as possible, aiming just to replace the interfaces with digital
equivalents while retaining the usage model. First, there is a shared global screen,
on which the digital notes can be placed or removed. This screen is expected to be
large enough to
accommodate several users standing in front of it at the same time, and that the notes
that are displayed on the wall are large enough to be read from a few feet away. To
interact with the wall, each user utilizes their cellphones to go to a specific URL in
their browser. This webpage asks for their name, and then they are free to begin. On
their phone, a user is first presented with buttons to create a new note and to pick
a note up off the global screen. On hitting the create note button, they are greeted
with an interface to allow them to create a note, choose its color, etc. Once they are
satisfied with the content, they can click the button at the bottom of the UI to place
the note. This activates their cursor on the global screen, which the user manipulates
via pointing to the location they want, at which point they may press another button
to place the note. When not creating a note, users may also activate their cursor on
the large screen to pick a note up off of it for editing it locally. When a user picks
a note up, it is removed from the large screen until the user places it back, thus
preventing anyone else from seeing or interacting with it. In addition to the MUIFOLD
interface to the use-case, we also allow users to utilize voice commands to accomplish
some of the state change operations, such as changing the note's color and deleting it.

Under the hood, for each created note, the system records information on who created it,
what color it is, and once on the large screen, its x,y coordinates. The four actions
available to users are (1) create a note, (2) delete a note, (3) place a note, and (4)
pick up a note. Finally, for each note that is created, it is given a unique numerical
ID that increments for each additional note added to the system.
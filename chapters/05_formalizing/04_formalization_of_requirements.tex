\section{Formalizing the Requirements for a CAIS}

It should be noted that a CAIS to be considered a truly ``intelligent''
room, it is not sufficient that the room be intelligent about, for
example, search queries over a domain $D$; the room should also be
intelligent about cognitive states of agents in the room and their
cognitive states towards $D$.

Despite there being a significant amount of work done in building
intelligent environments (of varying levels of intelligence;
\cite{coen_design_1998, brooks_intelligent_1997,chan_review_2008}), there
is no formalization of what constitutes an intelligent room and what
separates it from an intelligent agent.  Though \cite{coen_design_1998}
briefly differentiates an intelligent room from ubiquitous computing
based on the non-ubiquity of sensors in the former, there is not any
formal or rigorous discussion of what separates an intelligent room
from a mobile robot that roams around the room with an array of
sensors.  We offer below a sketch of informal requirements that an
immersive room should aim for.  Then we instantiate these requirements
using \CEC.

Assuming that these two conditions hold, we can use them to make determinations
about the properties of such systems, as well as to help distinguish the various
classes of intelligent rooms and their capabilities that have been historically created.
%\subsection{Formal Requirements for a CAIS}
%\label{sect:freqs}
From the information requirements, we look to translate them into a formalized 
version for a CAIS. The formal requirements are explained below:

    \begin{mdframed}[frametitle= Formal Requirements for $\mathcal{C}$ ,
      frametitlebackgroundcolor=gray!25, nobreak,
      linecolor=white,backgroundcolor=gray!10]
      Assume $\Gamma \vdash t < t + \Delta$
      \begin{enumerate}
      \item[$\mathbf{C}^f_1:$] It is common knowledge that, if an
        agent $x$ has a false belief, $\gamma$ informs the agent of
        the belief:\footnote{Please note that inference in \DCEC\ is
          non-monotonic as it includes the event calculus, which is
          non-monotonic.  If an agent $a$ believes $\phi$ based on
          prior information, adding new information can cause the
          agent to not believe $\phi$.}
         \begin{equation*}
          \common\color{gray!80}\left(\color{black}t, \left[\begin{aligned}
                &  \believes(\gamma, t, \phi) \land \believes(\gamma, t,
                \believes\big(x, t, \lnot \phi)\big)
                \\ & \hspace{50pt} \rightarrow \\
                & \hspace{30pt}\says(\gamma, x, t + \Delta, \phi)
              \end{aligned}\right]\color{gray!80}\right)\color{black}
        \end{equation*}

      \item[$\mathbf{C}^f_2:$] It is common knowledge that, if an
        agent $x$ has a missing belief, $\gamma$ informs the agent of
        the belief:
        \begin{equation*}
          \common\color{gray!80}\left(\color{black}t,\left[ \begin{aligned}
                &  \believes(\gamma, t, \phi) \land \believes(\gamma, t,
                \lnot \believes\big(x, t, \phi)\big)
                \\ &  \hspace{50pt}  \rightarrow \\
                & \hspace{30pt}\says(\gamma, x, t + \Delta, \phi)
              \end{aligned}\right]\color{gray!80}\right)\color{black}
        \end{equation*} \end{enumerate}
    \end{mdframed}
    
\begin{mdframed}[frametitle= Formal Requirements for $\mathcal{I}$ ,
  frametitlebackgroundcolor=gray!25, nobreak,
  linecolor=white,backgroundcolor=gray!10]
  \begin{enumerate}
  \item[$\mathbf{I}^f_1:$] It is common knowledge that at any
    point in time $t$ an agent $x$, different from the CAIS system
    $\gamma$, can observe events or conditions (fluents) only in
    its vicinity.
    \begin{equation*}
      \begin{aligned}
        \common\Bigg(\forall x, t, f:& (x \not = \gamma) \rightarrow \Big[\perceives\big(x, t, \holds(f, t)\big) \rightarrow
        \holds(\vicinity(x, f), t)\Big]\Bigg)\\
        \common\Bigg(\forall x, t, e:& (x \not = \gamma) \rightarrow \Big[\perceives\big(x, t, \happens(e, t)\big) \rightarrow
        \holds(\vicinity(x, e), t)\Big]\Bigg)\\
      \end{aligned}
    \end{equation*}

  \item[$\mathbf{I}^f_2:$] It is common knowledge that actions
    performed by the agent are in its vicinity.
    \begin{equation*}
      \common\Big(\forall x, \alpha:  \holds\big(\vicinity(x, \action(a, \alpha)), t\big)\Big)
    \end{equation*}
  \item[$\mathbf{I}^f_3:$] It is common knowledge that all events
    and fluents are perceived by $\gamma$.  This is represented by
    the four conditions below:
    \begin{equation*}
      \begin{aligned}
   &  (i) \hspace{10pt}  \common\Bigg(\forall  t, f:  \Big[\holds(f, t) \leftrightarrow \perceives\big(\gamma, t, \holds(f, t)\big) \Big]\Bigg)\\
   & (ii)  \hspace{10pt}   \common\Bigg(\forall  t, e: \Big[\happens(e, t)
        \leftrightarrow \perceives\big(\gamma, t, \happens(e,
        t)\big) \Big]\Bigg)\\
    &   (iii)  \hspace{10pt}   \common\Bigg(\forall  t, f:  \Big[\lnot \holds(f, t)
        \leftrightarrow \perceives\big(\gamma, t, \lnot \holds(f, t)\big) \Big]\Bigg)\\
    &   (iv) \hspace{10pt}    \common\Bigg(\forall  t, e:\Big[\lnot \happens(e, t)
        \leftrightarrow \perceives\big(\gamma, t, \lnot \happens(e, t)\big) \Big]\Bigg)
      \end{aligned}
    \end{equation*}
  \end{enumerate}
\end{mdframed}

Time is assumed to be discrete, as in the
discrete event calculus presented in \cite{mueller_commonsense_2014}.  There is a
background set of axioms and propositions $\Gamma(t)$ that is
operational at time $t$.  We have a fluent $\vicinity$ that tells us
whether an agent is in the vicinity of a fluent, event, or another
agent:

$$\vicinity: \Agent \times \Fluent \cup \Event \cup \Agent \rightarrow
\Fluent $$

Only events and fluents in the vicinity of an agent can be
observed by the agent.

\begin{comment}
For the cognitive condition $\mathcal{C}$ above, we have the following
concrete requirements that we implement in our system.  Later, we give
examples of these requirements in action.

  \begin{footnotesize}
    \begin{mdframed}[frametitle= Formal Requirements for $\mathcal{C}$ ,
      frametitlebackgroundcolor=gray!25, nobreak,
      linecolor=white,backgroundcolor=gray!10]
      Assume $\Gamma \vdash t < t + \Delta$
      \begin{enumerate}
      \item[$\mathbf{C}^f_1:$] It is common knowledge that, if an
        agent $x$ has a false belief, $\gamma$ informs the agent of
        the belief:\footnote{Please note that inference in \DCEC\ is
          non-monotonic as it includes the event calculus, which is
          non-monotonic.  If an agent $a$ believes $\phi$ based on
          prior information, adding new information can cause the
          agent to not believe $\phi$.}
         \begin{equation*}
          \common\color{gray!80}\left(\color{black}t, \left[\begin{aligned}
                &  \believes(\gamma, t, \phi) \land \believes(\gamma, t,
                \believes\big(x, t, \lnot \phi)\big)
                \\ & \hspace{50pt} \rightarrow \\
                & \hspace{30pt}\says(\gamma, x, t + \Delta, \phi)
              \end{aligned}\right]\color{gray!80}\right)\color{black}
        \end{equation*}

      \item[$\mathbf{C}^f_2:$] It is common knowledge that, if an
        agent $x$ has a missing belief, $\gamma$ informs the agent of
        the belief:
        \begin{equation*}
          \common\color{gray!80}\left(\color{black}t,\left[ \begin{aligned}
                &  \believes(\gamma, t, \phi) \land \believes(\gamma, t,
                \lnot \believes\big(x, t, \phi)\big)
                \\ &  \hspace{50pt}  \rightarrow \\
                & \hspace{30pt}\says(\gamma, x, t + \Delta, \phi)
              \end{aligned}\right]\color{gray!80}\right)\color{black}
        \end{equation*} \end{enumerate}
    \end{mdframed}
  \end{footnotesize}

  For the immersive condition $\mathcal{I}$ above, we have the
  following conditions:

  \begin{footnotesize}
    \begin{mdframed}[frametitle= Formal Requirements for $\mathcal{I}$ ,
      frametitlebackgroundcolor=gray!25, nobreak,
      linecolor=white,backgroundcolor=gray!10]
      \begin{enumerate}
      \item[$\mathbf{I}^f_1:$] It is common knowledge that at any
        point in time $t$ an agent $x$, different from the CAIS system
        $\gamma$, can observe events or conditions (fluents) only in
        its vicinity.
        \begin{equation*}
          \begin{aligned}
            \common\Bigg(\forall x, t, f:& (x \not = \gamma) \rightarrow \Big[\perceives\big(x, t, \holds(f, t)\big) \rightarrow
            \holds(\vicinity(x, f), t)\Big]\Bigg)\\
            \common\Bigg(\forall x, t, e:& (x \not = \gamma) \rightarrow \Big[\perceives\big(x, t, \happens(e, t)\big) \rightarrow
            \holds(\vicinity(x, e), t)\Big]\Bigg)\\
          \end{aligned}
        \end{equation*}

      \item[$\mathbf{I}^f_2:$] It is common knowledge that actions
        performed by the agent are in its vicinity.
        \begin{equation*}
          \common\Big(\forall x, \alpha:  \holds\big(\vicinity(x, \action(a, \alpha)), t\big)\Big)
        \end{equation*}
      \item[$\mathbf{I}^f_3:$] It is common knowledge that all events
        and fluents are perceived by $\gamma$.  This is represented by
        the four conditions below:
        \begin{equation*}
          \begin{aligned}
       &  (i) \hspace{10pt}  \common\Bigg(\forall  t, f:  \Big[\holds(f, t) \leftrightarrow \perceives\big(\gamma, t, \holds(f, t)\big) \Big]\Bigg)\\
       & (ii)  \hspace{10pt}   \common\Bigg(\forall  t, e: \Big[\happens(e, t)
            \leftrightarrow \perceives\big(\gamma, t, \happens(e,
            t)\big) \Big]\Bigg)\\
        &   (iii)  \hspace{10pt}   \common\Bigg(\forall  t, f:  \Big[\lnot \holds(f, t)
            \leftrightarrow \perceives\big(\gamma, t, \lnot \holds(f, t)\big) \Big]\Bigg)\\
        &   (iv) \hspace{10pt}    \common\Bigg(\forall  t, e:\Big[\lnot \happens(e, t)
            \leftrightarrow \perceives\big(\gamma, t, \lnot \happens(e, t)\big) \Big]\Bigg)
          \end{aligned}
        \end{equation*}
      \end{enumerate}
    \end{mdframed}
  \end{footnotesize}

\end{comment}
  %%% Local Variables:
  %%% mode: latex
  %%% TeX-master: "main"
  %%% End:

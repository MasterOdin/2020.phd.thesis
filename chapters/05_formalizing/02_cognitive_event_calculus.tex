\section{The Cognitive Event Calculus}

%TODO:
%   1. go through and expand CEC
%   2. remove duplicate tables of syntax/sorts
%   3. Add subsection on shadowprover

To capture the CAIS in a formal way, we employ the
\textbf{cognitive event calculus} (\CEC), a cognitive calculus and
intensional logic, which provides a subset of the \DCEC~\cite{nsg_sb_dde_ijcai}
\footnote{For a richer primer on the \DCEC, we invite users to view the
appendix of that work}. The \CEC\ is a multi-sorted quantified modal logic with a 
well-defined syntax and proof calculus. The use of sorts here can be thought of
as being analogous to a typed single-inheritance programming language. The
proof calculus of \CEC\ is based on natural deduction 
\cite{gentzen_investigations_1964} and includes all the introduction and
elimination rules for first-order logic, as well as inference schemata for
the modal operators and related structures discussed below. At its core, the
\CEC\ subsumes the event calculus~\cite{mueller_commonsense_2014} (EC), a
first-order calculus used for
modeling events and actions and their effects upon the world. The modal
operators that the \CEC\ adds on top of the EC are that
$\believes$elief, $\knows$nowledge, $\common$ommon Knowledge, $\perceives$erceiving,
$\says$aying, $\desires$esiring, and $\intends$ntending. At a surface level,
the \CEC\ may appear similar to readers to BDI logics~\cite{rao_modeling_1991},
but  differ in a number of key ways. The most significant is that BDI logics
utilizes \textit{possisble-worlds semantics} with uninstantiated inference
schema, but rather utilizes \textit{proof-theoretic sematics} as it is based
on natural deduction~\cite{gentzen_untersuchungen_1935}\footnote{Bringsjord and
Govindarajulu~\cite{bringsjord_given_2012} provide a deeper explanation for
the interested reader.}. Additionally, through the inclusion of operators
like perception, communication, and intensional operators, the \CEC\ aims
to facilitate ease of use within fields of computer vision and natural language.

\begin{table}
\begin{footnotesize}
\begin{center}
\begin{tabular}{lp{8cm}}
\toprule
\textbf{Sort}    & \textbf{Description} \\
\midrule
\type{Agent} & Human and non-human actors.  \\

\type{Moment} &  Time points. E.g., $t_i$, $birthday(son(jack))$ etc. \\

  \type{Event} & Used for events in the domain. \\
  \type{ActionType} & Abstract actions
                      instantiated by agents.\\
  \type{Action} & Events that occur
                  as actions by agents \\
  \type{Fluent} & Representing states of the world.\\
  \type{Formula} & Represents any arbitrary formula\\
  \bottomrule
\end{tabular}
\caption{Basic sorts in the language}
\label{table:sorts}
\end{center}
\end{footnotesize}
\end{table}

\begin{figure}[ht]
% frametitle=Signature (snippet), 
\begin{mdframed}[roundcorner=8pt, nobreak=true, frametitlealignment=\centering]
\begin{equation*}
 \begin{aligned}
    \mathit{S} &::= 
    \left\{\begin{aligned}
      & \Object \sep \Agent  \sep \ActionType \sep \Action \sqsubseteq
      \Event \sep \\ &\Moment  \sep \Fluent \sep \Numeric\\
    \end{aligned} \right.
    \\ 
    \hspace{-29pt}\mathit{f} &::= \left\{
    \begin{aligned}
      & \action: \Agent \times \ActionType \rightarrow \Action \\
      &  \initially: \Fluent \rightarrow \Boolean\\
      &  \holds: \Fluent \times \Moment \rightarrow \Boolean \\
      & \happens: \Event \times \Moment \rightarrow \Boolean \\
      & \clipped: \Moment \times \Fluent \times \Moment \rightarrow \Boolean \\
      & \initiates: \Event \times \Fluent \times \Moment \rightarrow \Boolean\\
%      & \terminates: \Event \times \Fluent \times \Moment \rightarrow \Boolean \\
      & \prior: \Moment \times \Moment \rightarrow \Boolean\\
    \end{aligned}\right.\\
        \mathit{t} &::=
    \begin{aligned}
      \mathit{x : S} \sep \mathit{c : S} \sep f(t_1,\ldots,t_n)
    \end{aligned}
    \\ 
    \mathit{\phi}&::= \left\{ 
    \begin{aligned}
     & 
        A(\ldots):\Boolean \sep 
        \neg \phi \sep
        \phi \land \psi \sep
        \phi \lor \psi \sep \\
    &
        \phi \lif \psi \sep
        \phi \liff \psi \sep
        \exists x \phi(x) \sep
        \forall x \phi(x)\\
     &
        \perceives (a,t,\phi)  \sep
        \knows(a,t,\phi) \sep
        \common(t,\phi) \sep
        \believes(a,t,\phi) \\
    &
        \says(a,t,\phi) \sep
        \says(a,b,t,\phi)  \sep
        \desires(a,t,\phi) \sep
        \intends(a,t,\phi) \\
      \end{aligned}\right.
  \end{aligned}
\end{equation*}
\end{mdframed}
\caption{\CEC\ Syntax}
\label{fig:cec_syntax}
\end{figure}

We know provide a brief primer on the event calculus and \CEC\ to make
accessible the following chapter. The sorts of our logic are shown in 
Table~\ref{table:sorts}. Most of the sorts shown come from the underlying EC,
but we add here three additional sorts: \Agent, \Action, and \ActionType.
The syntax of the \CEC\ is shown in Figure~\ref{fig:cec_syntax}. In our
syntax, the symbols of the event calculus are contained under $\mathit{f}$.
An important axiom that drives much of the event calculus is:
\vspace{-0.4cm}
\begin{equation*}
    \begin{aligned}
        \happens(e, t_{1}) \land \initiates(e, f, t_{1}) \land (t_{1} < t) \land \lnot\clipped(t_{1}, f, t) \rightarrow \holds(f, t)
    \end{aligned}
\end{equation*}

\noindent
which can be read as "some event $e$ \happens at time $t_{1}$ and that this $e$
\initiates our fluent $f$ at $t_1$ such that at some later time $t$, and assuming
that there has not been something causing $f$ to be \clipped (made false) between
$t_{1}$ and $t$, then $f$ holds at time $t$". From this axiom, we see that the
event calculus is non-monotonic, such that fluents can hold at some time, but
not hold at other times, and that this is determined by the events of the
system at large.

On top of this, as part of the allowed functions and relations to the system,
shown as $\mathit{\phi}$, we add the intensional operators which can be read as:
\perceives\ is an agent $a$ at time $t$ perceives $\mathit{\phi}$, \knows\
is an agent at time $t$ knows $\mathit{\phi}$, \common\ is at time $t$ that
$\mathit{\phi}$ is common knowledge to all agents, \believes\ is
an agent $a$ at time $t$ believes $\mathit{\phi}$ at time $t$, \says\ is
agent $a$ at time $t$ says $\mathit{\phi}$ or that agent $a$ at time $t$
says $\mathit{\phi}$ to agent $b$, \desires\ is agent $a$ at time $t$ desires
$\mathit{\phi}$, and \intends\ is agent $a$ at time $t$ intends $\mathit{\phi}$.
With our syntax established, we now move to the inference schemata, shown in
Figure~\ref{fig:cec_schemata}.

\begin{figure}[h]
% frametitle=Inference Schemata (snippet),
\begin{mdframed}[nobreak=true, roundcorner=8pt, frametitlealignment=\centering]
\begin{equation*}
\begin{aligned}
% &\mbox{Sample rules below. For more rules, see
%   \cite{akratic_robots_ieee_n}.}\\
  &\infer[{[R_{\knows}]}]{\knows(a,t_2,\phi)}{\knows(a,t_1,\Gamma), \ 
    \ \Gamma\vdash\phi, \ \ t_1 \leq t_2} \\ 
& \infer[{[R_{\believes}]}]{\believes(a,t_2,\phi)}{\believes(a,t_1,\Gamma), \ 
    \ \Gamma\vdash\phi, \ \ t_1 \leq t_2} \\
& \infer[{[R_1]}]{\common(t,\perceives(a,t,\phi) \lif\knows(a,t,\phi))}{}\\
&  \infer[{[R_2]}]{\common(t,\knows(a,t,\phi)
    \lif\believes(a,t,\phi))}{}\\
  &\infer[{[R_3]}]{\knows(a_1, t_1, \ldots
    \knows(a_n,t_n,\phi)\ldots)}{\common(t,\phi) \ t\leq t_1 \ldots t\leq
    t_n}\hspace{10pt}
  \infer[{[R_4]}]{\phi}{\knows(a,t,\phi)}\\
  & \infer[{[R_5]}]{\common(t,\knows(a,t_1,\phi_1\lif\phi_2))
    \lif \knows(a,t_2,\phi_1) \lif \knows(a,t_3,\phi_2)}{}\\
& \infer[{[R_6]}]{\common(t,\believes(a,t_1,\phi_1\lif\phi_2))
    \lif \believes(a,t_2,\phi_1) \lif \believes(a,t_3,\phi_2)}{}\\
& \infer[{[R_7]}]{\common(t,\common(t_1,\phi_1\lif\phi_2))
    \lif \common(t_2,\phi_1) \lif \common(t_3,\phi_2)}{} \\
& \infer[{[R_8]}]{\common(t, \forall x. \  \phi \lif \phi[x\mapsto
  t])}{} \\
& \infer[{[R_9]}]{\common(t,\phi_1 \liff \phi_2 \lif \neg
    \phi_2 \lif \neg \phi_1)}{}\\
& \infer[{[R_{10}]}] {\common(t,[\phi_1\land\ldots\land\phi_n\lif\phi]
  \lif [\phi_1\lif\ldots\lif\phi_n\lif\psi])}{}\\
% & \infer[{[R_{11a}]}]{\believes(a,t,\psi)}{\believes(a,t,\phi)\ \ \phi
%   \lif \psi}\
%\hspace{6pt} \infer[{[R_{11b}]}]{\believes(a,t,\psi \land \phi)}{\believes(a,t,\phi)\ \ \believes(a,t,\psi)}\\ 
& \infer[{[R_{12}]}]{\believes(h,t,\believes(s,t,\phi))}{\says(s,h,t,\phi)} \\
& \infer[{[R_{13}]}]{\perceives(a,t,\happens(\action(a^\ast,\alpha),t))}{\intends(a,t,\happens(\action(a^\ast,\alpha),t'))}
\end{aligned}
\end{equation*}
\end{mdframed}
%\end{scriptsize}
\caption{\CEC\ Inference Schemata}
\label{fig:cec_schemata}
\end{figure}

$R_\mathbf{K}$ and $R_\mathbf{B}$ state that for knowledge and belief, if an
agent knows or believes some formula $\Gamma$ at time $t_{1}$, and that $\Gamma$
entails some formula $\phi$, then at some equal or later time $t_{2}$, the
agent will know or belief the formula $\phi$. Under these two rules, it
establishes that knowledge and belief are closed under the inference system of
\CEC. $R_{1}$ and $R_{2}$ state that it is common knowledge that perception
leads to knowledge and that knowledge to belief respectively. $R_{3}$ states that
any formula that is common knowledge, all agents know that formula, and also know
that all other agents know that formula, iteratively. $R_{4}$ states that
knowledge  of a formula implies that that formula holds. Through the use of
$R_{3}$ and $R_{4}$, we can unwind any formula that is common knowledge, such
as for $R_1$, we could derive
$\perceives(a, t, \phi) \rightarrow \knows(a, t, \phi)$. $R_5$ to $R_{10}$
provide for a more restricted form of reasoning for propositions that are common
knowledge, unlike propositions that are known or believed.  $R_{12}$
states that if an agent $s$ communicates a proposition $\phi$ to $h$,
then $h$ believes that $s$ believes $\phi$. $R_{13}$ states that for for some
action that agent $a$ intends action $\alpha$, then they will perceive that they
perceive that action at a different time. For our work here, $R_{1}$ to $R_{4}$
and $R_{12}$ are most important for our formalization and reasoning efforts.
To utilize our logic as part of the CAIS, past the formalization, we utilize
an automated theory prover, \textsf{ShadowProver}, which is described in greater
detail in section~\ref{sec:automated_reasoner_planner}.

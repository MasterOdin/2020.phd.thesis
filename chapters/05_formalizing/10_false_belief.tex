\section{Solving False-Belief Task in CAIS}

As a demonstration of our formalization process, we utilize our CAIS to solve the
false-belief task~\cite{frith_theory_2005}. To accomplish this, we instantiate a very
elementary cognitive-polysolid world that involves only three blocks (though 

To illustrate the usage of the framework, we use a very elementary cognitive-polysolid world (though this
could scale upward to larger numbers of blocks).  
We have three blocks,
named $\ablock$, $\bblock$, and $\cblock$, which all start on the
table, shown to the participants on the center display. This is represented in the $\CEC$ as:


\begin{center}
\begin{tabular}{ c c }
    \holds(\on(\ablock, \ctable), 0) & 
    \holds(\clear(\ablock), 0)\\
    \holds(\on(\bblock, \ctable), 0) &     
    \holds(\clear(\bblock), 0)\\
    \holds(\on(\cblock, \ctable), 0) & 
    \holds(\clear(\cblock), 0)
\end{tabular}
\end{center}

There are only two human agents, $\humana$ and $\humanb$, who have
knowledge about how the cognitive-polysolid world works.  Using this
instantiation, we give the room two tasks to demonstrate its theory of
mind as required by the constraints specified above.  For both tasks,
we will use the same sequence of events to configure the world:

\begin{enumerate}
  \item{\humana\ and \humanb\ enter the room}
  \item{\humana\ moves block \ablock\ onto block \bblock}
  \item{\humanb\ adds the goal of block \cblock\ on block \bblock}
  \item{\humanb\ leaves the room}
  \item{\humana\ moves block \ablock\ to the table}
  \item{\humana\ removes the goal for block \cblock\ and adds the goal of block \ablock\ on block \cblock}
  \item{\humana\ moves block \ablock\ onto block \cblock}
  \item{\humanb\ returns to the room}
  \item{\humanb\ tries to move \ablock\ to the table.}
\end{enumerate}

For this simulation, all events and fluents inside the room are
considered to be in the vicinity of agents within the room, and none
of the events and fluents within the room are considered to be in the
vicinity of agents outside the room when they happen or hold.

\begin{figure}
  \begin{center}
    \begin{minipage}[b]{0.25\textwidth}
      \begin{flushleft}   
        \begin{footnotesize} \underline{\humana's goals and beliefs}\\
          \vspace{4pt}
          \textbf{goals}: $On(A,C)$\\
          \vspace{-4pt}
          \textbf{belief}:
        \end{footnotesize}
      \end{flushleft}    
      \begin{center} \begin{tikzpicture}[auto centering, background rectangle/.style={fill=white}, show background rectangle]
          \node[style=block] (C) {$C$};
          \node[style=empty] (E) [above=of C] {};
          \node[style=block] (B) [left=of C, left=.5cm of C] {$B$};
          \node[style=block] (A) [left=of B, left =.5 cm of B] {$A$};
        \end{tikzpicture}
      \end{center}
    \end{minipage}
   \hspace{50pt}
    \begin{minipage}[b]{0.25\textwidth}
      \begin{flushleft} 
        \begin{footnotesize} \underline{\humanb's  goals and beliefs}\\                          
          \vspace{4pt}
          \textbf{goals:} $On(C,B)$\\
          \vspace{-4pt}
          \textbf{belief}:
        \end{footnotesize}
      \end{flushleft}   \begin{tikzpicture}[auto centering, background rectangle/.style={fill=gray!25}, show background rectangle]
        \node[style=block] (B) {$B$};
        \node[style=emptyA] (E) [above=of C] {};
        \node[style=block] (A) [above=of B] {$A$};
        \node[style=block] (C) [right=of B, right=.5cm of B] {$C$};
        \node[style=empty] (F) [right=of C] {};
      \end{tikzpicture}
    \end{minipage}
  \end{center}
  \caption{Depiction of agents' mental states with the shaded portion indicating that \humanb\ is not in the room.}
  \label{fig:mom-example}
\end{figure}
 
For the first portion of this task, we consider the world between
steps 5 and 6.  At this point, we wish to see where the room believes
the blocks are, as well as where it believes that \humana\ and
\humanb\ think the blocks are, focusing primarily on block A.  We ask
the machine three questions, translating them into the \CEC:
\emph{``Where does the CAIS/$\humana$/$\humanb$ believe block
  $\ablock$ is?''}.  We translate this question into three sentences
in the \CEC\ which we can then pass down to \textsf{ShadowProver} to
answer.
% For each question, we convert it to an
% "exists" check and then parses the returned proof to find what
% statement unified with it to give us the location of where the agent
% believes block A is in the world. The exists statements for the
% questions are of the form
% $\exists{x}\believes(a, t, \holds(\on(\ablock, x), t))$ (where $a$
% represents the agent under question)
For the first two questions, both the AI and the agent $\humana$ are
in the room and can perceive where the block is, and thus have
knowledge of its location.  $\humanb$ left the room at step 4 and
missed the block being moved at step 5.  Therefore, his knowledge of
where the block is remains at what it was when he was in the room. Through
\textsf{ShadowProver}, we obtain an answer to the above question
of where each agent believes the block is (we show a visual representation
of these answers in Figure~\ref{fig:mom-example}): 

\vspace{-0.1in}
\begin{footnotesize}
\begin{equation*}
\begin{aligned}\
{\believes(\cir, t, \holds(\on(\ablock, \ctable), t)), \believes(\humana, t, \holds(\on(\ablock, \ctable), t)),
\believes(\humanb, t, \holds(\on(\ablock, \bblock), t))}
\end{aligned}
\end{equation*}
\end{footnotesize}
\vspace{-0.15in}
\section{Formalizing the Modules of our CAIS}

In this section, we now provide formalizations for the modules that exist within our
CAIS that exist across all use-cases. Going back to the discussion in 
Chapter~\ref{chap:technology}, we assume that a given CAIS implementation will minimally
feature some ability to point at content and an ability to input intents into the system,
such as through voice, typing natural language sentences, or clicking on content. We now
quickly go through these mechanisms here for usage within the \CEC.

\subsection{Pointing}

We first focus on formalizing pointing in our system. While agents may point at anything 
within the room, we focus principally on agents pointing at content on a screen. To 
accomplish this,as was discussed earlier we utilize the Reagent system to allow us to
capture content  shown on arbitrary webpages. For each piece of meaningful content shown on 
the screen, we  assume that it can be expressed by some fluent. An agent then points at the 
fluent, and that information is captured by our CAIS. From this, we provide our
formalization of the point action:

\begin{equation*}
\begin{aligned}
  \point: \Agent \times \Object \rightarrow \ActionType
\end{aligned}
\end{equation*}

For this to operate without flooding our system within point actions, we only consider
``meaningful pointing'' events. These events are ones in which a user points at a piece
of content for a minimum of a second. This is to avoid registering ``interim pointing''
events, which occur as an agent moves their pointing across the screen to its final
destination.

\subsection{Inputted Intents}


\section{Future Work}

We now discuss some promising avenues of future work. We examine two main
thrusts here, within each there are a number of possible extensions and
improvements that may be investigated. The first thrust concerns our
formalization efforts and how we may build upon it. As stated in
Chapter~\ref{chap:formalizing}, we purposely used the \CEC\ to form our
basis, leaving ourselves open to various extensions to it to deal with
more advanced scenarios. Perhaps the biggest extension is to handle the
potentially noisiness of incoming data from sensors and feeding it into
our system. Within this work, we assumed that sensors were providing
perfect information with perfect confidence. In reality, many of the input
mechanisms operate with degrees of confidence on their result. For example,
for each classification that is provided by the conversation-worker, the
Watson Assistant service provides a confidence of its interpretation. For
certain actions, lower allowed confidence levels are fine, such as in the
scenarios presented here, but in more high stakes environments,
the CAIS may need to conduct certain actions only with high confidence and
some with low. To handle this, we look to work by Govindarajulu and
Bringsjord where they incorporate strength factors into a cognitive
calculus~\cite{govindarajulu_strength_2017}. Strength factors can be viewed
as a formalization of part of Chisholm’s rational
belief-fixation~\cite{theory.of.knowledge3.chisholm}. Within the work,
the belief operator is amended such that it has a coefficient, $\sigma$,
such that it ranges from one to five, where beliefs are labelled as (1)
acceptable, (2) some presumption in favor, (3) beyond reasonable doubt,
(4) evident, and (5) certain. An added benefit of this approach is that
the presentation of these strength factors may be easier for understanding
from the layperson versus raw probability values, which people may find
difficult to parse~\cite{kaye_can_1991}.

Another avenue of interest for our formalization efforts is in dealing with 
ethical issues and matters of privacy. When utilizing these collaborative spaces,
as information about specific users grows, a user would have a reasonable
belief that the system will not offer sensitive information to their
colleagues without prior approval. The ``Deontic Cognitive Event
Calculus'' (\DCEC) is an attractive logic to handle these cases, adding in the
deontic operator of obligation on-top of the \CEC. Indeed, Bringsjord et al.
proposes a ``Tenancular AI'' shows the usefulness of the \DCEC\ with regards
to these concerns~\cite{bringsjord_tentacular_2018}. Additionally, the \DCEC\
has been used to model the ethical principals of ``Doctrine of Double
Effect''~\cite{govindarajulu_automating_2017} and ``Doctrine of Triple
Effect''~\cite{peveler_towards_2018}.

The other thrust concerns our technology. First, while we are pleased with
our preliminary results of MUIFOLD and its effectiveness, from it were
identified a number of limitations. The most principle of is in making the
system more robust for how users hold their phones. As identified in
Section~\ref{chap04:limitations} however, we find that users would prefer to
hold their phone at an angle, and then point around that angle. However, our
initial implementation assumed that users were holding their phone flat, which
allowed for a simplification of detecting angle changes only around pitch and
yaw. Implementing roll into the equation would expand usability as we look
to extend our technology into more real-world applications. In that vein,
the evaluation here was preliminary in nature, and a more robust user study
is still required for full evaluation of our technology, especially as it
compares to other mechanisms of relative pointing (e.g. Bluetooth clickers)
and absolute pointers (e.g. Wii remote) both in our laboratory testing, as
well as in real-world usage. Perhaps even more importantly is that the
technology is deployed to real-world usage wherein we may discover additional
issues or strengths only discovered when users engage the technology in a
more natural environment than the Fitts' pointing test.

Finally, the technology and formalization presented here focuses on
co-located environments. This focus was a purposeful one due to the
inherit differences between co-located and distributed that we were
not prepared to address here. The first step towards this would be
working on extending the CAIS implementation we presented here to
a fully distributed set-up, wherein each participant is running their
own transcript-worker, speaker-worker, display-worker, etc., and that
the content to be outputted may be the same, or different per user to
fit their needs, such as the display-workers may have different content
open per user, or the content may be arranged differently. All of this
is to be accounted for within the technology stack, but also must be
incorporated into formalization efforts herein. However, to accomplish this,
our first steps would be on deploying the technology to users for a
given scenario, and perform baseline user studies to understand empirically
real-world usage. From this, we could then better conceptualize and deal
with formalization of the environment, and then bring to bear the fruits
of that formalization effort, such as in intent resolution and plan recognition.
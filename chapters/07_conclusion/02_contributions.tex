\section{Contributions of the Dissertation, Met}

\begin{enumerate}
    \item \checkmark\ Create a high-level but rigorous and fertile definition for what
        constitutes a CAIS, separating it from a physical robot.

    \begin{itemize}
        \item[] Within this dissertation, we have provided definition for what makes a system
        \textit{cognitive} ($\mathcal{C}$) and what makes it \textit{immersive} ($\mathcal{I}$).
        From these definitions, we can see what an idealized version of a CAIS may look like,
        and indeed, how it a localized physical robot may be able to achieve elements of both
        $\mathcal{C}$ and $\mathcal{I}$, it is not possible to achieve them both fully, or even
        the version presented within this thesis.
    \end{itemize}
  
    \item \checkmark\ Define a framework for how one approaches building a CAIS.
    \begin{itemize}
        \item[] We presented a high-level conceptual framework in
        Section~\ref{sec:cais_architecture} for building out these sorts of systems. Due to the
        high potential number of configurations these systems may be employed into, the framework advocates for
        a modular event-driven system, wherein fusion of sensor data, such as might be needed
        happens downstream from any initial parsing of that sensor. In this purpose, the system
        can handle the loss of a sensor or input mechanism, and that may require the user to
        interact with the system in a different manner, but even there, we can still achieve our
        definition of a CAIS.
    \end{itemize}
    
    \item \checkmark\ Create a real-world implementation of a CAIS, demonstrating at a
  novel level that it:
        \begin{enumerate}
            \item \checkmark\ adapts to a number of potential input mechanisms;
            \item \checkmark\ operates at a true multi-modal and multi-user level; and
            \item \checkmark\ is capable of interacting with and understanding
              third-party content.
        \end{enumerate}
        
    \begin{itemize}
        \item[] Within Chapters~\ref{chap:technology},\ref{chap:reagent},\ref{chap:muifold},
        we created a real world implementation of our CAIS along with several new novel
        technologies to achieve our sub-goals above. From \textit{Reagent}, we demonstrated
        a novel technology that is capable of providing us a functional hook into web-based
        content within the room, understanding what is shown on the page as well as capturing
        and driving user interaction withe content. Additionally, because of its design, Reagent
        can be applied not just to content created specifically for the CAIS, but for most web
        content one might wish to open. Rising from this, we present the \textit{Virtual Mouse API}, wherein
        it allows us to provide all users of our system with their own mouse cursor and ability
        to interact with pages simultaneously with one another. From MUIFOLD, we provide
        a novel framework for building out a UI for cellphones to be used within the CAIS, both
        as a generic pointing device for all users, but also for being able to surface deeper
        interactions with sites that supports it, that would be otherwise impossible on traditional
        pointing devices such as HTC Vive and Kinect cameras. Additionally, due to the modular design
        of all modulars of our system as dictated by our framework, pieces of the work have been
        successfully deployed
        elsewhere~\cite{allen_rensselaer_2019,chabot_collaborative_2020,divekar_humaine_2020}.
    \end{itemize}

    \item \checkmark\ Create a formal definition of a CAIS' implementation and
        its capabilities.
        
    \begin{itemize}
        \item[] In Chapter~\ref{chap:formalizing}, we utilize the \CEC\ drive a formalization of
        our high-level definition. In this way, we provide in exact terms and formulae of the
        cognitive and immersive features discussed in this dissertation. Following this, we also
        provide formalization of the components we utilize in the CAIS, tying together how our
        implementation works and relates to the formalization. Finally, because we have provided
        in exact terms our definitions in this fashion, we were able to derive the property of
        \textit{Expectation of Usefulness}, further separating out a CAIS from a physical robot.
        The formalization, as well as our defined properties and formuale, are to our knowledge
        novel compared to prior work in the space of intelligent or smart rooms.
    \end{itemize}

    \item Utilize the above items to handle tasks of reasoning, planning, and plan recognition
        in and involving a CAIS.
    
    \begin{itemize}
        \item[] Finally, given our implementation and formalization above, we were able to utilize
        these to accomplish tasks in reasoning, planning, and plan recognition, operating at a
        ``theory of mind'' level throughout. Not only that, due to the mechanisms employed, for all
        decisions made by our system, they are fully explainable via a backing proof. Each of these
        demonstrations were deployed and demonstrated in the real-world (albeit against toy domains),
        and that we have shown in highly rigorous terms the power of our cognitive and immersive systems.
    \end{itemize}
\end{enumerate}
\section{Defining Planning and Plan Recognition}

Before continuing, we explicitly define here the components of planning. We base our
work off of the definitions by Ramírez and Geffner~\cite{ramirez_plan_2009}.
Within this work, we utilize the STRIPS planning problem formalism, where each problem is the
tuple $P = \langle F, I, A, G \rangle$. For this tuple, $F$ stands for the set of
fluents that can define the state of world, $I \subseteq F$ is the initial state of
fluents, $G \subseteq F$ is the goal state of fluents, and $A$ is the set of actions
available to an agent. For each action $a \subseteq A$, it has a set
of preconditions $pre(a)$, add effects $add(a)$, and deletion effects $del(e)$. For a given state
$s$, it is made up of a number of fluents that all hold true. After an action $a$ is executed
by the system, we define the successor state of $s$ as
$\delta(a, s) = ((s \backslash del(a)) \cup add(a))$. An example definition of an
action from the blocks world domain is shown in Figure~\ref{fig:strips} in the STRIPS-style
language we employ for our planner. A full listing of all actions for the CAIS is available in Appendix~\ref{apdx:strips}. Completion of a planning problem is then the computation of
an action sequence, $\pi = a_{1}, a_{2}, ..., a_{m}$, where each
$a_{i} \subseteq A$, and that following the action sequence starting at $I$
will arrive at $G$.

\begin{figure}[ht]
\begin{lstlisting}[]
    (define-action stack [(Block ?x) (Block ?y)]
        {
            :preconditions [
                (on ?x Table)
                (clear ?x)
                (clear ?y)
            ]
            :additions     [(on ?x ?y)]
            :deletions     [
                (clear ?y)
                (on ?x Table)
            ]
        }
    )
\end{lstlisting}
\caption{Stack action defined in STRIPS-style language.}
\label{fig:strips}
\end{figure}

As stated above, plan recognition on the other hand is planning in reverse,
wherein we now have only observations of agents to compare against a given plan.
Any computed $\pi$, be it from a plan library or constructed on the fly, then
must satisfy a given observation sequence.

From this, we can more regularly define the plan recognition problem. A plan recognition
problem can be defined as the tuple $R = \langle P, \mathcal{G}, O \rangle$, where
$P = \langle F, I, A \rangle$ as the planning domain (using the above terms),
$\mathcal{G}$ is the set of possible goals $G \subseteq F$ and
$O = o_{1}, o_{2}, ...., o_{n}$ is the observation sequence by agents where each
$o_{i}$ being an action in $A$. The observation sequence is then a subset
of a given action sequence wherein each observation matches into the action
sequence, where they follow the same ordering, but that there may be some missing
actions. For example, for the given action sequence $\pi = {a,b,c,d,e}$, then
the sequences ${a,c,e}$ and ${b,d,e}$ are valid, while ${c,a,e}$ is not.
\section{Plan Recognition for Course Correction}

We finish this chapter with an examination of using plan recognition within our CAIS.
For this, we re-use our scenario from Section~\ref{sect:false_belief}. To re-iterate
the situation, we have two human agents, \humana\ and \humanb, and they are working
in the blocks world domain. Same as before, there are three blocks that are shown
on the screen and they all initially are on the table. This information is
shown below:

\begin{center}
\begin{tabular}{ c c }
    \holds(\on(\ablock, \ctable), 0) &
    \holds(\clear(\ablock), 0)\\
    \holds(\on(\bblock, \ctable), 0) &
    \holds(\clear(\bblock), 0)\\
    \holds(\on(\cblock, \ctable), 0) &
    \holds(\clear(\cblock), 0)
\end{tabular}
\end{center}

This time through, we consider a modified sequence from above, incorporating goals
into the system. On adding a goal to our system, the orchestrator is now set to also
validate all actions taken by agents are in line with the current goals of the system.
Any action that is not valid is then blocked by the orchestrator for proceeding. The
new action sequence is as follows:

\begin{center}
\begin{tabular}{l | l | l}
    $t$ & Event in Natural Language & Event in \CEC\ \\
    \hline
    1 & \humana\ enters and registers with the CAIS & \happens(\register(\humana), 1) \\
    2 & \humanb\ enters and registers with the CAIS & \happens(\register(\humanb), 2) \\
    3 & \humana\ stacks block \ablock\ onto block \bblock\ & \happens(\stack(\ablock, \bblock), 3) \\
    4 & \humanb\ adds goal of block \cblock\ on block \bblock & \happens(\setGoal(\on(\cblock, \bblock)), 4) \\
    5 & \humanb\ deregisters and leaves the CAIS & \happens(\deregister(\humanb), 5) \\
    6 & \humana\ unstacks block \ablock\ from block \bblock\ & \happens(\unstack(\ablock, \bblock), 6) \\
    7 & \humana\ removes the goal of block \cblock\ on block \bblock\ & \happens(\removeGoal(\on(\cblock, \bblock)), 7) \\
    8 & \humana\ adds the goal of block \ablock\ on block \cblock & \happens(\setGoal(\on(\ablock, \cblock)), 8) \\
    9 & \humana\ stacks block \ablock\ onto block \cblock & \happens(\stack(\ablock, \cblock), 9) \\
    10 & \humanb\ re-enters and registers with CAIS & \happens(\register(\humanb), 10) \\
\end{tabular}\label{table:plan_recognition_actions}
\end{center}

\begin{center}
\begin{tabular}{l | l | l}
    $t$ & Added Fluent & Removed Fluents \\
    \hline
    1 & \holds(\inCAIS(\humana), 1) & \\
    2 & \holds(\inCAIS(\humanb), 2) & \\
    3 & \holds(\on(\ablock, \bblock)), 3) & \makecell[l]{
        \holds(\on(\ablock, \ctable), 2) \\
        \holds(\clear(\bblock), 2) \\
    }\\
    4 & \holds(\goal(\on(\cblock, \bblock)), 4) & \\
    5 & & \holds(\inCAIS(\humanb), 4) \\
    6 & \makecell[l]{
        \holds(\on(\ablock, \ctable), 6) \\
        \holds(\clear(\bblock), 6) \\
    } & \makecell[l]{
        \holds(\on(\ablock, \bblock), 5) \\
    } \\
    7 & & \holds(\goal(\on(\cblock, \bblock)), 6) \\
    8 & \holds(\goal(\on(\ablock, \cblock)), 8) & \\
    9 & \holds(\on(\ablock, \cblock), 9) & \makecell[l]{
        \holds(\on(\ablock, \ctable), 8) \\
        \holds(\clear(\cblock), 8) \\
    } \\
    10 & \holds(\inCAIS(\humanb), 10) & \\
\end{tabular}\label{table:plan_recognition_fluents}
\end{center}

Our diligent \humanb, having returned to the CAIS, sees the state of the world
has moved away from the goal that he set in step 4. He, seeking to achieve that
old goal he still believes is active in the system, attempts to issue the command
``Unstack block A from block C'', as he seeks to stack block C onto block B. Our
orchestrator in this case refuses to complete that action as to do so would
move the world away from our goal state of $\on(\ablock, \cblock)$. On refusal
of executing the move, the orchestrator launches into a fallback of plan recognition
over \humanb. The first thing done by the system is to collect the potential goals
that an agent may be following. In this, the system scans through its knowledge base
to find goal addition events that happened within the vicinity of \humanb, giving
us the goals of $\on(\cblock, \bblock)$ and $\on(\ablock, \cblock)$. Next, as per
our definition, the system generates the sequence of actions that would take the
state of the world at the start of step 9 to achieve that goal, giving us an
action sequence. The system finally then takes the action of step 9 and determines
if it within either plan's action sequence. In this case, it finds it as part of
the sequence for the goal of $\on(\cblock, \bblock)$. Taking this information, the
CAIS then scans through its and \humanb's knowledge bases to determine where things
may have gone awry. In this, it determines that \humanb\ did not perceive the goal
state change ate step 6. A summary of this information is then all outputted to the
agents, wherein it shows the agent what plan they were attempting to follow, the plan
that they should have been following, and why they have a difference of information.
This catches the agent up, and their knowledge base is now assumed in regards equal
to that of the room for this sequence.

In addition to utilizing plan recognition above, our CAIS also provides an instantiation
of our property of \textit{Expectation of Usefulness}. This property is
instantiated with the event $e$ being setting a new goal:
$e \equiv \setGoal(\on(\ablock, \cblock)))$:

\begin{footnotesize}
\begin{equation*}
\begin{aligned}
&\perceives\big(\humana, t_8, \happens\big(\setGoal(\on(\ablock,\cblock)), t_{8}\big)\big) \land \perceives\Big(\humana, t_8, \lnot \holds\big(\vicinity(\humanb, setGoal (On(A,C))), t_8\big)\Big) \\
& \hspace{90pt} \rightarrow \believes\Big(\humana, t_{11}, \says\big(\gamma, \humanb, t_{11},
\happens(\setGoal(\on(\ablock,\cblock)), t_8)\big)\Big)
\end{aligned}
\end{equation*}
\end{footnotesize}

\section{Automated Reasoner and Planner}\label{sec:automated_reasoner_planner}

\subsection{ShadowProver}

To allow our CAIS utilize the reasoning as defined in 
chapter~\ref{chap:formalizing}, we employ an automated theorem prover,
\textsf{ShadowProver}~\cite{govindarajulu_shadowprover_2018}\footnote{The
  prover is available in both Java and Common Lisp and can be obtained
  at: \url{https://github.com/naveensundarg/prover}. The underlying
  first-order prover is SNARK, available at:
  \url{http://www.ai.sri.com/~stickel/snark.html}.}. Traditionally,
first-order modal logic theorem provers
that can work with arbitrary inference schemata are built upon first-order
theorem provers. They achieve the reduction to first-order logic via two
methods. In the first method, modal operators are simply represented by
first-order predicates as in the example shown above. This approach is the
fastest but can quickly lead to well-known inconsistencies as demonstrated.
In the second method, the entire proof theory is implemented intricately in 
first-order logic, and the reasoning is carried out within first-order
logic. Here, the first-order theorem prover simply functions as a
general-purpose declarative programming system. This approach, while
accurate, can be excruciatingly slow.

\textsf{ShadowProver} utilizes a different approach to the above to achieve
speed without sacrificing consistency in the system. At the core of this
approach is a technique named ``shadowing'', wherein ShadowProver applies a
syntactic operation to convert any modal formula (or a set of formulae) $\phi$
to a non-modal formula $\mathsf{shadow}[\phi]$ by replacing the atomic modal
sub-formulae with propositional atoms. Through tis, ShadowProver is then able
to alternate between applying the high-level modal inference schemata and calling
out to a dedicated first order prover, until a proof has been found or not for the
given axioms and goal statement.

\subsection{Spectra}

Planning for our CAIS is handled by \textsf{Spectra}~\cite{govindarajulu_spectra_2018},
and planner based on an \emph{extension} of the STRIPS-style planning language discussed
above. At the core of \textsf{Spectra} is a backing of \textsf{ShadowProver} for resolving
actions. Through this, we can extend our planning formalism in several important
ways, namely that we allow arbitrary formulae of the \CEC\ in the definition
of states, actions, and goals. For instance, valid states and goals can
include: \emph{``No three blocks on the table should be of the same
color.''}  and \emph{``Jack believes that Jill believes there is one block on
the table.''}.
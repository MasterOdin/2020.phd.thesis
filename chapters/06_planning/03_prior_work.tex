\section{Prior Work}

Planning, and more specifically plan recognition, is a rich field of
research that has a broad range of prior work that we draw from. For planning,
one branch of work is ``deductive planning'', wherein given a goal and a series 
of axioms,
the resulting proof gives a plan structure~\cite{green_application_1969}. This
approach is attractive as it allows the usage of first-order logic to define
our world, and the state transitions. Further work was done to handle concepts
of conditional branching, plan composition, and 
recursion~\cite{metzing_plan_1989,biundo_deductive_1992,rosenschein_plan_1981}.
However, through the regular use of FOL, we see a larger potential of ending
up with side-effects of actions, which can be undesirable for planning. An
alternative concept is through the usage of STRIPS-style 
planners~\cite{fikes_strips_1971}. Within these planners, each action is
defined via a complete structure with pre-conditions and post-conditions for
each action. This approach is attractive as it provides a useful mechanism for
dealing with the framing problem~\cite{mccarthy_philosophical_1969}. However,
these planners generally utilize a predicate calculus that lacks the full
expressiveness of first-order logic, not to mention higher order logics like
the \CEC. While STRIPS-styles planners work well across many domains, they
rely on deterministic planning. Extending the concept, conditional nonlinear
planners~\cite{peot_conditional_1992} and partial 
planners~\cite{pryor_planning_1996}
provide a mechanism for dealing with non-deterministic plans through if-else
like structures. However, this comes at a cost of increased complexity and
combinatorial explosion of states without ensuring some limitation on the
expressivity~\cite{rintanen_constructing_1999}. Given that we aim for increased 
expressivity, we do not consider these approaches herein. As such, we also do
not concern ourselves with other extensions to classical planners such as 
probabilistic 
planners~\cite{boutilier_decision-theoretic_1999,kaelbling_planning_1998}.
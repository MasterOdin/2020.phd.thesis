%%%%%%%%%%%%%%%%%%%%%%%%%%%%%%%%%%%%%%%%%%%%%%%%%%%%%%%%%%%%%%%%%%% 
%                                                                 %
%                            ABSTRACT                             %
%                                                                 %
%%%%%%%%%%%%%%%%%%%%%%%%%%%%%%%%%%%%%%%%%%%%%%%%%%%%%%%%%%%%%%%%%%% 
 
\specialhead{ABSTRACT}

As computational power has continued to increase, and high fidelity sensors and
large-scale displays have become commonplace, so too has the
desire for
%% MATT:  I don't think ubiquitous admits of degrees.
%% I would strike 'more'.  //S
%% Rephrased to "commonplace", does that work for you? // M
systems that can better fuse artificial intelligence and
human-computer interaction.  On the AI side of things, there is a
demand for artificial intelligent agents to operate at a cognitive
level that permits them to reason over an agent's beliefs, knowledge,
communications, goals, etc.  On the HCI side, these systems must be
multi-modal, able to combine speech, gestures, and various interfaces
to allow a diverse range of interactions to these systems.  Within
this fusion, users may develop reasonable and productive expectations
regarding the capabilities of these systems, and these capabilities
can then be applied to a diverse range of domains and content, in
service of said users.

To address these concerns, in this dissertation we introduce a
framework for building what we term \textit{cognitive and immersive
systems} (CAIS).  Within our approach, and to handle the demands
above, we emphasize techniques that are well-formalized and that
operate at a ``theory of mind'' level.  To accomplish this, we look to
undergird our system with a formal language that has a high level of
expressivity.  For this, we turn to the \textit{cognitive event
calculus} (\CEC), a multi-operator multi-sorted quantified modal
logic, and a matching high-expressivity automated reasoner and
planner.  Stemming from this we can establish a formalized set of
principles by which these systems operate; and these systems can
operate within scenarios that require not only reasoning about the
physical world, but also the cognitive states of our agents within the
CAIS.  To validate our approach and demonstrate its effectiveness, we
provide a real-world open-source implementation of a CAIS, and utilize
it on tasks of planning and plan recognition.
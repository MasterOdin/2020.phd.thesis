%%%%%%%%%%%%%%%%%%%%%%%%%%%%%%%%%%%%%%%%%%%%%%%%%%%%%%%%%%%%%%%%%%% 
%                                                                 %
%                            ABSTRACT                             %
%                                                                 %
%%%%%%%%%%%%%%%%%%%%%%%%%%%%%%%%%%%%%%%%%%%%%%%%%%%%%%%%%%%%%%%%%%% 
 
\specialhead{ABSTRACT}

As computational power has continued to increase, and sensors and large
scale displays have become more ubiquitous, so too has the desire for
systems that can better fuse artificial intelligence and human-computer
interaction. On the AI side of things, there is a demand for it to
operate at a cognitive level, reasoning over an agent's beliefs,
knowledge, communications, goals, etc. On the HCI side, these systems
must be multi-modal, being able to combine speech, gestures,
and various interfaces to allow a diverse range of interactions to
these systems. Within this fusion, users may develop expectations on
the capabilities of these systems, and that these capabilities can
be applied to a diverse range of domains and content.

To address these concerns, within this dissertation, we introduce a
framework for building what we term \textit{cognitive and immersive systems}
(CAIS). Within our approach, and to handle the demands above, we
emphasis an approach is well formalized and that operates at a
``theory of mind'' level. To accomplish this, we look to back our
system with a language that has a high level of expressivity. For this,
we turn to the \textit{cognitive event calculus} (\CEC), a multi-sorted
quantified modal logic, and a matching high-expressivity automated
reasoner and planner. Stemming from this we can establish a formalized
set of prinicples by which these systems operate, and can operate
within scenarios that require not only reasoning about the physical world,
but also the cognitive states of our agents within the CAIS. To validate
our approach and demonstrate its effectiveness, we provide a real-world
open-source implementation of a CAIS, and utilize it on tasks of
planning and plan recognition.
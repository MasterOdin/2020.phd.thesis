\section{Summary}

In this chapter, we laid out and demonstrated MUIFOLD, a novel
framework for building out mobile interfaces for interaction
with our CAIS, namely in the context of large displays. The principal
contribution of this framework, compared to prior work, is its
ability to work across a wide range of websites, both first party
and third party, out of the box, while providing easy extension points
for building out custom UIs for a given site. This was demonstrated
in a sample use-case for intelligence analysts that utilized a mix
of third-party and first-party content, and which some screens
utilizing a custom UI and others not. However, of equal importance is
that our approach by utilizing the builtin sensors of a cellphone
and relying on a method of relative ray-casting allowed our technology
to be used immediately in new environments without the costly or
difficult steps of calibration as seen in other approaches that utilized
more precise external sensors or devices. We also provided an initial
user study to evaluate the basics of the technology, namely on the
reliability of the pointing, and ability of the users to be able to select
and manipulate content on the screen. While our results were below
that of prior work, we identified several ways in which we may boost
our results in follow-up work. However, perhaps more importantly is that
what we lose in pointing accuracy, is more than made up on in the increased
amount of interactions that are possible by utilizing the phone's screen
for displaying content to the user from the large screen.
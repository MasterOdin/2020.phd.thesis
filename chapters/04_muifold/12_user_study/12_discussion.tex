\subsection{Discussion}

From our results, we see that while our efforts here are usable, there is
room for improvement, especially as compared to results reported by
Zhang~\cite{zhang_combining_2017}. However, while scoring a lower accuracy
overall, users were still able to complete the task in a time relatively
similar to prior work, and also maintain a compelling reaction time during the
task. From our post-experiment interviews and observations, it seems like the
biggest issues face by our users was on overall experience with these types of 
systems and the required phone position during the task. For the latter,
we had one participant who indicated high prior experience with smart rooms,
with all devices sans cellphones, and was then able during the test to score
above 90\% on all trials with a time equal to or below the other participants.
This would indicate that given additional practice time, the participants would
get further used to the technology and how it functions, and we could expect that
the accuracy would improve. However, a quicker gain would be on relaxing the
constraints on how users had to hold their devices. As indicated above, users
had to hold the phones flat in their hands, and that only changes in the yaw
and pitch of the phone were considered for moving the on-screen cursor. As stated
above, we did not initially tell users how to hold their phones, allowing us to
observe that almost universally all users wanted to hold their phones at least at
a slight degree of roll off the x-axis, with at least one user wanting to hold it
at an angle between 45$\deg$ and 90$\deg$. Improving this functionality would
enable users to hold their phones at a more comfortable and natural angle, which
we hypothesis would lead to improvements in accuracy and time taken on the tasks.
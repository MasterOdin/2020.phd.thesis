\section{Discussion of Limitations}\label{chap04:limitations}

In this section, we provide some lessons learned about certain
aspects of the technology from our preliminary user study and
sample use-case.

\textbf{Location and Size of UI Buttons} One of the biggest design challenges
was dealing with placement of
buttons for the generic UI, as well as general guidelines for the
developers of webview specific UIs. The principle goal was that
users of the system, after some practice, would be able to use
their phones and the buttons on it without having to look down
at it. This was accomplished in three ways. First, buttons we expect a
user to use without looking down are centered around the middle
$\frac{1}{3}$ of the screen. This corresponds to the average rotation
of our user's thumb when naturally holding their devices. Second,
each button was given a large dimensions to make it harder to ``fat
finger'' a wrong button without looking. Thirdly, we provide haptic
feedback through quick phone vibrations whenever a user clicks a
button, allowing them to know that they hit it. Unfortunately, we
found during implementation that haptic feedback was only available
to android users.

\textbf{Permissions} To utilize many of the underlying sensors or
capabilities of the phone in a web context, permissions are frequently
asked, such as if attempting to use location, use the camera, etc. However,
due to historical reasons, this is not true for the DeviceOrientation
API on most devices which we rely on for determining where the phone
is pointing at. Many users found this somewhat disconcerting. We
found that users would feel more comfortable if we showed a fake
permissions prompt asking users if we could utilize the gyroscope
and accelerometer. Interestingly, while developing the framework,
iOS 13 was released which actually does require users to give
permission to use the sensors, but this is done through system
settings, which some users found a bit confusing if not provided
step-by-step instructions. However, in using React, we can utilize
the ReactNative library to generate native apps that do not have
these issues, at the cost of requiring the user to download and install
the app.

\textbf{Providing ``Fluid'' UIs} Among our 10 subjects, very few shared
the same phone model, and each had a different amount of
available screen real estate. To handle such differences,
there's a paradigm of ``fluid'' layout according to which the content
dimensions are based on percentages of available screen real estate. This
allowed us to deal with the size differences. Unfortunately, when
opening the keyboard on both iOS and android, the screen size as
reported to the page changes, causing the content to become messed up
in location and size. To overcome this, when loading our client
application, we implement a two-stage loading process. In the
first phase, we load the content relative to the size of the screen.
In the second phase, we sweep over all elements and provide them
fixed sizes and positions. Luckily, it is easy to detect screen
rotation to re-draw all elements as necessary, but we do not need to
worry about the screen size otherwise changing.

\textbf{Browser and Device Incompatibilities} The biggest challenge
we faced was in slight browser incompatibilities across a range of
devices. The first is that each browser provides a different frame
of reference for the coordinate system used by the DeviceOrientation
API. For example, Firefox and Chrome browsers on the same device would
consistently report angles that were 20$\deg$ apart. Additionally,
refreshing the open page might cause the sensors to ``reset'' and report
a completely new angle. Finally, on certain devices, the browsers
would report the orientation in an opposite direction from the way
the phone was rotating, causing the cursor to go in the opposite
direction. There did not happen on specific browsers regardless of
device, so the solution was to provide a user with a setting to
flip orientation tracking to correct it.

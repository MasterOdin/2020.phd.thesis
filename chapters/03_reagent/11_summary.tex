\section{Summary}

In this chapter, we have explained and demonstrated the Reagent system. Reagent
is capable of taking advantage of commonly-occurring structural motifs and
human-friendly tagging such as tooltips. This in-turn makes it easier for
developers to create cognitive application that support natural voice-based
interactions with pre-existing webpages containing structured data such as tables
and plots. Backing from this, we also show how Reagent can be used to help
drive a virtual mouse interface that through how it functions can be fully multi-user
through pointing and gestures, improving upon limitations of prior work. This interface
is intended to allow hooking up to a wide range of technologies demonstrated in
the prior work, including Kinect cameras~\cite{allen_rensselaer_2019},
Oblong wands~\cite{kephart_embodied_2019}, and HTC Vive controllers~\cite{zhao_immersive_2018},
as well as for allowing us to explore additional types of input mechanisms, like the
MUIFOLD system covered in the following chapter.

To demonstrate Reagent, we used it to build an ontology in OWL through two unrelated websites,
as well as provide an experimental section to showcase how the system handles
parsing of several popular websites showcasing generalizability. As part of this demonstration,
we showed how Reagent can readily learn the vocabulary of a domain by asking questions when it
does not understand the user's or the webpage's terminology.

An future avenue of interest would be in utilization of this technology in ontology building.
For this work, we assumed that we were starting on a blank slate. In a real-world application,
one could assume that there may be a pre-existing ontology, or that it would be possible to
import a relevant one to the domain in question. To accomplish this, we might consider utilization
of platforms such as DBPedia~\cite{auer2007} or ConceptRDF~\cite{najmi_conceptrdf:_2016}, both
of which readily translate into OWL. Through these, we could pre-populate an ontology with
entities and some attributes and as a user browses pages, the system could automatically, or
ask the user first, append to the imported entities building a larger and more comprehensive super-set.